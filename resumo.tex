%\begin{resumo} 
%
%\end{resumo}
%
%\section{Introdução}
%
%\section{Conclusão}

O trabalho se concentra no tema de segurança da informação, com foco especial na autenticação, um processo fundamental que vai além da simples proteção de dados e serviços. A autenticação é a base para a verificação da identidade do usuário, sendo crucial em um cenário digital onde a integridade das informações é constantemente ameaçada. Entender e aprimorar esses mecanismos é vital para garantir que apenas as pessoas autorizadas tenham acesso aos recursos, ao mesmo tempo que se aprofunda nos desafios e nas soluções para proteger essa etapa vital contra ameaças cada vez mais sofisticadas.

Um dos problemas mais comuns no processo de autenticação é o uso de senhas fracas ou repetidas. Muitos usuários criam combinações simples e as reutilizam em diferentes serviços, aumentando o risco de ataques de força bruta e de vazamentos em massa. Outro grande desafio é a dependência excessiva de um único fator de autenticação, como as senhas, que se mostraram insuficientes diante de técnicas como phishing, keylogging e ataques de engenharia social. Por isso, a adoção da autenticação multifator (MFA) tem se tornado essencial, embora também apresente limitações, como a interceptação de códigos por SMS ou a dificuldade de implementação em certos contextos.

Como objeto de estudo, o trabalho vai abordar os Mecanismos e Protocolos de Autenticação. Isso inclui desde métodos tradicionais como senhas e biometria, até soluções mais modernas como tokens de segurança e sistemas de autenticação adaptativa. Esses mecanismos são os componentes técnicos que materializam os conceitos do domínio de conhecimento, e o objetivo é aprimorar sua eficácia e usabilidade.
Nesse contexto, a pesquisa é guiada pela seguinte pergunta central:\textit{ "Como os diversos métodos de autenticação influenciam na mitigação de ataques de engenharia social e quais seus desdobramentos na segurança em plataformas digitais?"}